\documentclass{acm_proc_article-sp}
\usepackage{cite}

\begin{document}
\title{Semantic Analysis of Epilepsy Patient Discharge Records}
\subtitle{A Project for EECS600: The Semantic Web}
\numberofauthors{2}
\author{
\alignauthor Ian Dimayuga \\
    \email{ian.dimayuga@case.edu}\\
\alignauthor Tom Dooner \\
    \email{tom.dooner@case.edu}\\
}
\maketitle
\begin{abstract}
\end{abstract}

\section{Introduction}
% From the proposal report, I guess?

\section{Background}
The broader field of applying Natural Language Processing (NLP) to Electronic Health Records
is well-discussed in academic journals. For example, a Clinical Data Architecture has been
proposed~\cite{CDA} which promises to structure electronic health records. Unfortunately, however,
decades of health records exist digitally in an unstructured format. Lacking structure, it is
difficult to write applications which can analyze any broad collection of records. Researchers
have found it prudent to consider using NLP technology to detect medical errors, correlate data, 
and draw conclusions on vast quantities of data~\cite{friedman}.

Prior work exists in other NLP-related fields, such as evaluating how semantically similar 
two suggested medical terms are~\cite{Pedersen2007288} and how to analyze patient
discharge summaries to extract keywords and key relationships from the content of the
reports~\cite{soderland}. Perhaps the most relevant prior work is Uzuner et al., in 2006,
showed with some success that smoking status can be inferred from medical discharge 
records~\cite{Uzuner200814}.

While our goals are not to make as black-and-white a decision as Uzuner et al., we hope to take 
advantage of research contained in these referenced articles to produce a useful tool which will
help medical professionals draw conclusions by querying more-structured data.

\section{Project Goals}
We hope to achieve the following goals with our system:
\begin{itemize}
	\item Allow records to be easily added to the database.
	\item Allow users to search easily on basic criteria (date of birth, gender,
substance use) as well as more semantically deep criteria (age at first seizure,
seizure type, type of seizure).
    \item Store records in a semantically useful format.
	\item Employ a medical thesaurus so users can search for similar criteria easily.
\end{itemize}

\subsection{Significance}
The national Institute of Medicine has specifically noted NLP as a catalyst for a new
standard of medical care~\cite{friedman}. New functionality can be offered that is currently
inpractical due to human limitations on data access speed and correlation.

We hope that our application of NLP to the domain of patient summaries will afford the
web application users (medical professionals) with some never-before-seen insights.
Although our application will be operating on a small corpus of data with a small set 
of fields, in theory, semantically describing medicine records has wide-reaching 
significance in medicine.

\subsection{Prior Work}
The broader field of applying Natural Language Processing (NLP) to Electronic Health Records
is well-discussed in academic journals. For example, a Clinical Data Architecture has been
proposed~\cite{CDA} which promises to structure electronic health records. Unfortunately, however,
decades of health records exist digitally in an unstructured format. Lacking structure, it is
difficult to write applications which can analyze any broad collection of records. Researchers
have found it prudent to consider using NLP technology to detect medical errors, correlate data, 
and draw conclusions on vast quantities of data~\cite{friedman}.

Prior work exists in other NLP-related fields, such as evaluating how semantically similar 
two suggested medical terms are~\cite{Pedersen2007288} and how to analyze patient
discharge summaries to extract keywords and key relationships from the content of the
reports~\cite{soderland}. Perhaps the most relevant prior work is Uzuner et al., in 2006,
showed with some success that smoking status can be inferred from medical discharge 
records~\cite{Uzuner200814}.

While our goals are not to make as black-and-white a decision as Uzuner et al., we hope to take 
advantage of research contained in these referenced articles to produce a useful tool which will
help medical professionals draw conclusions by querying more-structured data.

\section{Methodology}
\subsection{Text Extraction}
% OCR, pdftotext -layout
When an epileptic patient is discharged from care, a Patient Discharge Summary is
filled out, detailing the patient's history, symptoms, the tests administered, as
well as diagnoses and recommended treatement. The majority of the documents are
presented in natural English language, while certain standard portions are in a
more structured format, with well-defined attributes and values. Finally, some
reports include a graphical representation of EEG data, which we excluded from analysis.
The body of data is provided as PDF scans, without text data. This necessitated the use
of Optical Character Recognition (OCR) to decipher the text from the PDF images.
After performing OCR, we used the pdftotext utility with the layout-preserving option
to extract the text data into program-readable files.

\subsection{Cleanup}
% also JSON formatting
\subsection{Tokenization}
% sections, sentences, words
The tokenization process was performed in three phases. The first phase was section
recognition. Luckily, many sections in Patient Discharge Summaries have headers presented
in large capital letters, which are easy to recognize. Furthermore, the majority of
section titles are the same throughout all Summaries. The second phase was to partition
each section into sentences. This was done using the punkt tokenizer from NLTK,
which splits cleanly and intelligently along sentence-delimiting punctuation.
Third and finally, we tokenized each sentence into words.
\subsection{Tagging}
\subsection{Chunking}
\subsection{Entity Recognition}
\section{Future Work}
% augment our grammar with a more general grammar
% study a larger dataset
\section{References}
\bibliography{citations}{}
\bibliographystyle{plain}
\end{document}
