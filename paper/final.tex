\documentclass{acm_proc_article-sp}
\usepackage{cite}
\usepackage{hyperref}
\hyphenation{ep-i-lep-sy}

\begin{document}
\title{Semantic Analysis of Epilepsy Patient Discharge Records}
\subtitle{A Project for EECS600: The Semantic Web}
\numberofauthors{2}
\author{
\alignauthor Ian Dimayuga \\
    \email{ian.dimayuga@case.edu}\\
\alignauthor Tom Dooner \\
    \email{tom.dooner@case.edu}\\
}
\maketitle
\begin{abstract}
We outline an application built using the Natural Language Toolkit\cite{nltk} which extracts meaningful information
from epilepsy patient discharge records. This information, designed to have high precision and recall, can be further
analyzed to determine trends across collections of records. Furthermore, we develop a simple web application frontend
to allow easy access to the backend parsing script.
\end{abstract}

\section{Introduction}

\section{Background}
The broader field of applying Natural Language Processing (NLP) to Electronic Health Records
is well-discussed in academic journals. For example, a Clinical Data Architecture has been
proposed~\cite{CDA} which promises to structure electronic health records. Unfortunately, however,
decades of health records exist digitally in an unstructured format. Lacking structure, it is
difficult to write applications which can analyze any broad collection of records. Researchers
have found it prudent to consider using NLP technology to detect medical errors, correlate data, 
and draw conclusions on vast quantities of data~\cite{friedman}.

Prior work exists in other NLP-related fields, such as evaluating how semantically similar 
two suggested medical terms are~\cite{Pedersen2007288} and how to analyze patient
discharge summaries to extract keywords and key relationships from the content of the
reports~\cite{soderland}. Perhaps the most relevant prior work is Uzuner et al., in 2006,
showed with some success that smoking status can be inferred from medical discharge 
records~\cite{Uzuner200814}.

While our goals are not to make as black-and-white a decision as Uzuner et al., we hope to take 
advantage of research contained in these referenced articles to produce a useful tool which will
help medical professionals draw conclusions by querying more-structured data.

\section{Project Goals}
Our project, as proposed, sought to accomplish the following goals.
\subsection{One Grammar, Multiple Files}
Using the same methodology, we want our technique to correctly determine attributes from a wide selection
of patient discharge reports.

Due to the short nature of this project, we were unable to obtain more than nine usable discharge reports,
so we made our goal to use these nine documents as a proof-of-concept for an extensible approach which
could apply to more discharge reports if they become available.

\subsection{Consistent Data Formatting}
In the patient discharge summaries, data is represented in plain text. While there are methods of
searching through plaintext, far more opportunities await if the data can be represented in a 
structured, meaningful format.

We seek to facilitate this improved data formatting by providing a consistent output format.
\subsection{Easy Search}
The formatted output should be easy to search. If you want to search for a captured attribute,
you should be able to in the structured output.

\section{Significance}
The national Institute of Medicine has specifically noted NLP as a catalyst for a new
standard of medical care~\cite{friedman}. New functionality can be offered that is currently
impractical due to human limitations on data access speed and correlation.

We hope that our application of NLP to the domain of patient summaries will afford the
web application users (medical professionals) with some never-before-seen insights.
Although our application will be operating on a small corpus of data with a small set 
of fields, in theory, semantically describing medicine records has wide-reaching 
significance in medicine.

\section{Prior Work}
The broader field of applying Natural Language Processing (NLP) to Electronic Health Records
is well-discussed in academic journals. For example, a Clinical Data Architecture has been
proposed~\cite{CDA} which promises to structure electronic health records. Unfortunately, however,
decades of health records exist digitally in an unstructured format. Lacking structure, it is
difficult to write applications which can analyze any broad collection of records. Researchers
have found it prudent to consider using NLP technology to detect medical errors, correlate data, 
and draw conclusions on vast quantities of data~\cite{friedman}.

Prior work exists in other NLP-related fields, such as evaluating how semantically similar 
two suggested medical terms are~\cite{Pedersen2007288} and how to analyze patient
discharge summaries to extract keywords and key relationships from the content of the
reports~\cite{soderland}. Perhaps the most relevant prior work is Uzuner et al., in 2006,
showed with some success that smoking status can be inferred from medical discharge 
records~\cite{Uzuner200814}.

While our goals are not to make as black-and-white a decision as Uzuner et al., we hope to take 
advantage of research contained in these referenced articles to produce a useful tool which will
help medical professionals draw conclusions by querying more-structured data.

\section{Methodology}
\subsection{Text Extraction}
% OCR, pdftotext -layout
When an epileptic patient is discharged from care, a Patient Discharge Summary is
filled out, detailing the patient's history, symptoms, the tests administered, as
well as diagnoses and recommended treatment. The majority of the documents are
presented in natural English language, while certain standard portions are in a
more structured format, with well-defined attributes and values. Finally, some
reports include a graphical representation of EEG data, which we excluded from analysis.
The body of data is provided as PDF scans, without text data. This necessitated the use
of Optical Character Recognition (OCR) to decipher the text from the PDF images.
After performing OCR, we used the pdftotext utility with the layout-preserving option
to extract the text data into program-readable files.

\subsection{Cleanup}
After extracting the text from the PDF documents, we remained with the complete raw text from the PDF files.
Unfortunately, this raw text includes a lot of things which might mess up later Natural Language Processing --
namely whitespace and the string \texttt{--De-identified--}, which replaces all personal data in our datasets.

Unfortunately, due to the human imperfect nature of data cleanup, the \texttt{--De-identified--} text does not remain
intact and is often spread amongst multiple words, split with spaces, and otherwise altered. Thus, after extracting
the text, we run it through a custom-written Ruby script which has a good success rate at removing instances of 
\texttt{--De-identified--} in its various forms.

Another benefit we get by preprocessing the text is the ability to break the file into sections. For instance,
all the major sections are titled with a line of all-captial letters. Using this, we can na{\"i}vely break the text
into sections which can be run individually afterwards.

To store these sections, the clean script outputs a JSON representation of a (key, value) object whose keys contain the
section heading and the values contain lists of line contents under the given key. This is read in by the parsing
script.

\subsection{Tokenization}
% sections, sentences, words
%                      words, words, words.
The tokenization process was performed in three phases. The first phase was section
recognition. Luckily, many sections in Patient Discharge Summaries have headers presented
in large capital letters, which are easy to recognize. Furthermore, the majority of
section titles are the same throughout all Summaries. The second phase was to partition
each section into sentences. This was done using the punkt tokenizer from NLTK,
which splits cleanly and intelligently along sentence-delimiting punctuation.
Third and finally, we tokenized each sentence into words.
\subsection{Tagging}
\subsection{Chunking}
Once each token has been tagged, we used a custom-built grammar to parse and chunk
the text. The majority of the productions dealt specifically with finding and parsing
sentences about prescription regimens, episodic semiology, comorbidity, and epileptogenic
zones. Some of this data was presented in structured list format, making the task
slightly easier for some cases. Unfortunately, this is the phase during which our
small dataset provided the greatest limitation. In the future, it will be necessary
to use both a larger dataset and a more general grammar to glean further semantic
data from the Discharge Summaries.
\subsection{Entity Recognition}
\section{Future Work}
% augment our grammar with a more general grammar
% study a larger dataset
\section{References}
\bibliography{citations}{}
\bibliographystyle{plain}
\end{document}
