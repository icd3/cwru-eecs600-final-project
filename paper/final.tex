\documentclass{acm_proc_article-sp}
\usepackage{cite}

\begin{document}
\title{Semantic Analysis of Epilepsy Patient Discharge Records}
\subtitle{A Project for EECS600: The Semantic Web}
\numberofauthors{2}
\author{
\alignauthor Ian Dimayuga \\
    \email{ian.dimayuga@case.edu}\\
\alignauthor Tom Dooner \\
    \email{tom.dooner@case.edu}\\
}
\maketitle
\begin{abstract}
\end{abstract}

\section{Introduction}
% From the proposal report, I guess?

\section{Background}
The broader field of applying Natural Language Processing (NLP) to Electronic Health Records
is well-discussed in academic journals. For example, a Clinical Data Architecture has been
proposed~\cite{CDA} which promises to structure electronic health records. Unfortunately, however,
decades of health records exist digitally in an unstructured format. Lacking structure, it is
difficult to write applications which can analyze any broad collection of records. Researchers
have found it prudent to consider using NLP technology to detect medical errors, correlate data, 
and draw conclusions on vast quantities of data~\cite{friedman}.

Prior work exists in other NLP-related fields, such as evaluating how semantically similar 
two suggested medical terms are~\cite{Pedersen2007288} and how to analyze patient
discharge summaries to extract keywords and key relationships from the content of the
reports~\cite{soderland}. Perhaps the most relevant prior work is Uzuner et al., in 2006,
showed with some success that smoking status can be inferred from medical discharge 
records~\cite{Uzuner200814}.

While our goals are not to make as black-and-white a decision as Uzuner et al., we hope to take 
advantage of research contained in these referenced articles to produce a useful tool which will
help medical professionals draw conclusions by querying more-structured data.

\section{Project Goals}
We hope to achieve the following goals with our system:
\begin{itemize}
	\item Allow records to be easily added to the database.
	\item Allow users to search easily on basic criteria (date of birth, gender,
substance use) as well as more semantically deep criteria (age at first seizure,
seizure type, type of seizure).
    \item Store records in a semantically useful format.
	\item Employ a medical thesaurus so users can search for similar criteria easily.
\end{itemize}

\subsection{Significance}
The national Institute of Medicine has specifically noted NLP as a catalyst for a new
standard of medical care~\cite{friedman}. New functionality can be offered that is currently
inpractical due to human limitations on data access speed and correlation.

We hope that our application of NLP to the domain of patient summaries will afford the
web application users (medical professionals) with some never-before-seen insights.
Although our application will be operating on a small corpus of data with a small set 
of fields, in theory, semantically describing medicine records has wide-reaching 
significance in medicine.

\subsection{Prior Work}
The broader field of applying Natural Language Processing (NLP) to Electronic Health Records
is well-discussed in academic journals. For example, a Clinical Data Architecture has been
proposed~\cite{CDA} which promises to structure electronic health records. Unfortunately, however,
decades of health records exist digitally in an unstructured format. Lacking structure, it is
difficult to write applications which can analyze any broad collection of records. Researchers
have found it prudent to consider using NLP technology to detect medical errors, correlate data, 
and draw conclusions on vast quantities of data~\cite{friedman}.

Prior work exists in other NLP-related fields, such as evaluating how semantically similar 
two suggested medical terms are~\cite{Pedersen2007288} and how to analyze patient
discharge summaries to extract keywords and key relationships from the content of the
reports~\cite{soderland}. Perhaps the most relevant prior work is Uzuner et al., in 2006,
showed with some success that smoking status can be inferred from medical discharge 
records~\cite{Uzuner200814}.

While our goals are not to make as black-and-white a decision as Uzuner et al., we hope to take 
advantage of research contained in these referenced articles to produce a useful tool which will
help medical professionals draw conclusions by querying more-structured data.

\section{Methodology}
\subsection{Extraction}
% OCR, pdftotext -layout
\subsection{Cleanup}
After extracting the text from the PDF documents, we remained with the complete raw text from the PDF files.
Unfortunately, this raw text includes a lot of things which might mess up later Natural Language Processing --
namely whitespace and the string \texttt{--De-identified--}, which replaces all personal data in our datasets.

Unfortunately, due to the human imperfect nature of data cleanup, the \texttt{--De-identified--} text does not remain
intact and is often spread amongst multiple words, split with spaces, and otherwise altered. Thus, after extracting
the text, we run it through a custom-written Ruby script which has a good success rate at removing instances of 
\texttt{--De-identified--} in its various forms.

Another benefit we get by preprocessing the text is the ability to break the file into sections. For instance,
all the major sections are titled with a line of all-captial letters. Using this, we can na{\"i}vely break the text
into sections which can be run individually afterwards.

To store these sections, the clean script outputs a JSON representation of a (key, value) object whose keys contain the
section heading and the values contain lists of line contents under the given key. This is read in by the parsing
script.

\subsection{Tokenization}
% sections, sentences, words
\subsection{Tagging}
\subsection{Chunking}
\subsection{Entity Recognition}
\section{Future Work}
% augment our grammar with a more general grammar
% study a larger dataset
\section{References}
\bibliography{citations}{}
\bibliographystyle{plain}
\end{document}
