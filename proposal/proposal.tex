\documentclass[12pt]{article}
\usepackage{fullpage}
\usepackage[left=1in,top=1in,right=1in,bottom=1in]{geometry}
\usepackage{titlesec}
\usepackage{enumerate}
\usepackage{amsmath}
\usepackage{amsthm}
\usepackage{paralist}
\usepackage{times}
\usepackage{cite}
\usepackage{hyperref} %for the URLs
\titleformat{\subsubsection}[runin]{}{}{}{}[]
\renewcommand{\labelenumi}{\alph{enumi}.}

\title{Semantic Analysis of Epilepsy-Related Patient Discharge Summaries}
\author{Ian Dimayuga, Tom Dooner \\icd3@case.edu, ted27@case.edu}
\begin{document}
\maketitle

\section{Background}
As defined by the NIH, Epilepsy is a brain disorder manifested in repeated seizures
over time.~\cite{nih-epilepsy} When patients have particularly bad seizures, they
are admitted to a hospital for monitoring. Upon admission to the hospital's Epilepsy
Monitoring Unit, various patient data is collected -- data such as Name, Sex, DOB,
and Age. The patient is also asked for some information regarding their recent seizure
episode such as zone, type, and origin.

This information, as well as the information collected from monitoring the patient
with EEGs during the hospital stay, is summarized in a discharge summary report.
These discharge reports are in a non-standard form and the data concerning background
information is in a natural-language form, making the data difficult to quickly
search.

We seek to codify and automatically determine the semantic relationships inherent
in patient discharge summaries to facilitate easy recall.

\subsection{Related Work}
The broader field of applying Natural Language Processing (NLP) to Electronic Health Records
is well-discussed in academic journals. Indeed, a Clinical Data Architecture has been
proposed~\cite{CDA} which promises to structure electronic health records. Unfortunately,
decades of health records exist digitally but in an unstructured format, so it is prudent
to consider using NLP technology to detect medical errors, correlate data, and draw
conclusions on vast quantities of data~\cite{friedman}.

Prior work exists in other NLP-related fields, such as evaluating how semantically similar 
two suggested medical terms are~\cite{Pedersen2007288} and how to analyze patient
discharge summaries to extract keywords and key relationships from the content of the
reports~\cite{soderland}. Perhaps the most relevant prior work is Uzuner et al., in 2006,
showed with some success that smoking status can be inferred from medical discharge 
records~\cite{Uzuner200814}.

While our goals are not as black-and-white as Uzuner et al., we hope to take advantage of
research contained in these referenced articles to produce a useful tool which will
help medical professionals draw conclusions by querying more-structured data.

\subsection{Significance}


\section{Approach}
The ultimate goal of the project is to produce a search-based web application.
Users will pose queries against an index of Patient Discharge Summaries, and the
application will output a subset of summaries deemed relevant based on a semantic
analysis of the data. Furthermore, the application will continually update the
knowledge base in the background as more data is added.

\subsection{Raw Data and OCR}
When an epileptic patient is discharged from care, a Patient Discharge Summary is
filled out, detailing the patient's history, symptoms, the tests administered, as
well as diagnoses and recommended treatement. The majority of the documents are
presented in natural English language, while certain standard portions are in a
more structured format, with well-defined attributes and values. Finally, some
reports include a graphical representation of EEG data, which will not be indexed.
The body of data is provided as PDF scans, without text data. This necessitates the use
of Optical Character Recognition (OCR) to decipher the text from the PDF images.

\subsection{Parsing and Entity Recognition}
Once the text data is acquired, we will use the Natural Language ToolKit (NLTK)
for the Python language to analyze the text syntactically to provide a structured
representation of the data. Once parsed, the tokens will be analyzed using specific
medical knowledge to determine the relevant entities found in the documents, and the
attribute values associated with them. Furthermore, the form portions of the documents
will be analyzed separately, since they are too structured to qualify as "natural
language."

\subsection{Indexing}
Each Discharge Summary will be associated with attributes and values based on the
entities recognized within the text. There will be a standard set of attributes
deemed important to all epilepsy-related Discharge Summaries. The standard set will
be determined partially from the medical knowledge base, and partially through an
analysis of the Discharge Summaries themselves for commonly reported information.
Every document entry will include these ancillary attributes, along with other
relevant information the document will contain. The documents will then be indexed
based on these attribute-value pairs. This indexing will take place in the background
dynamically as more Discharge Summaries are submitted to the knowledge base, and
the index along with the set of standard attributes will be updated dynamically.

\subsection{Search Application}
The website will present the user with search criteria by which he/she may find
the desired discharge summaries. These search criteria will be informed by the 
previously mentioned standard set of attributes. Numeric attributes such as dates
may be expressed as ranges. There will also be a search criterion for miscellaneous
information, which will utilize a thesaurus to find indexed attributes that do not
fall under the set of ancillary information. The user will then submit the search
query and retrieve a list of Discharge Summaries.

\section{Timeline}

\section{References}
\bibliography{citations}{}
\bibliographystyle{plain}


\end{document}
